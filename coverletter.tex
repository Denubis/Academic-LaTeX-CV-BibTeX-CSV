%!TEX TS-program = xelatex
%!TEX encoding = UTF-8 Unicode
% Awesome CV LaTeX Template for Cover Letter
%
% This template has been downloaded from:
% https://github.com/posquit0/Awesome-CV
%
% Authors:
% Claud D. Park <posquit0.bj@gmail.com>
% Lars Richter <mail@ayeks.de>
% Brian Ballsun-Stanton (brian.ballsun-stanton@mq.edu.au)
%
% Template license:
% CC-BY-SA 4.0 International
%


%-------------------------------------------------------------------------------
% CONFIGURATIONS
%-------------------------------------------------------------------------------
% A4 paper size by default, use 'letterpaper' for US letter
\documentclass[11pt, a4paper]{awesome-cv}

% Configure page margins with geometry
\geometry{left=1.4cm, top=.8cm, right=1.4cm, bottom=1.8cm, footskip=.5cm}

% Specify the location of the included fonts
\fontdir[fonts/]

% Color for highlights
% Awesome Colors: awesome-emerald, awesome-skyblue, awesome-red, awesome-pink, awesome-orange
%         awesome-nephritis, awesome-concrete, awesome-darknight
\colorlet{awesome}{awesome-red}
% Uncomment if you would like to specify your own color
% \definecolor{awesome}{HTML}{CA63A8}

% Colors for text
% Uncomment if you would like to specify your own color
% \definecolor{darktext}{HTML}{414141}
% \definecolor{text}{HTML}{333333}
% \definecolor{graytext}{HTML}{5D5D5D}
% \definecolor{lighttext}{HTML}{999999}

% Set false if you don't want to highlight section with awesome color
\setbool{acvSectionColorHighlight}{false}

% If you would like to change the social information separator from a pipe (|) to something else
\renewcommand{\acvHeaderSocialSep}{\quad\textbar\quad}


%-------------------------------------------------------------------------------
%	PERSONAL INFORMATION
%	Comment any of the lines below if they are not required
%-------------------------------------------------------------------------------


\name{Dr}{Brian}{Ballsun-Stanton}
\position{Solutions Architect (Digital Humanities)}


\address{Faculty of Arts, Macquarie University, NSW 2109}

\mobile{(+61) 479 178 749}


% \address{First Address \\ Second Address}
% \mobile{(+61) 111 111 111}

\email{brian.ballsun-stanton@mq.edu.au}
\orcid{0000-0003-4932-7912}
\github{denubis}
\homepage{https://osf.io/dza9b/}
% \gitlab{gitlab-id}
% \stackoverflow{SO-id}{SO-name}
% \twitter{@twit}
% \skype{skype-id}
% \reddit{reddit-id}
% \medium{madium-id}
\googlescholar{gc0PEWQAAAAJ}{}
%% \firstname and \lastname will be used
% \googlescholar{googlescholar-id}{}
% \extrainfo{extra informations}

%-------------------------------------------------------------------------------
%	LETTER INFORMATION
%	All of the below lines must be filled out
%-------------------------------------------------------------------------------
% The company being applied to
\recipient
 {Professor Bates Gill}
 {Macquarie University\\25 Wally's Walk\\Macquarie Park NSW 2109}
% The date on the letter, default is the date of compilation
\letterdate{\today}
% The title of the letter
\lettertitle{Job Application for Senior Lecturer in Cyber Security Analysis}
% How the letter is opened
\letteropening{Dear Professor Gill,}
% How the letter is closed
\letterclosing{With Warm Regards,}
% Any enclosures with the letter
%\letterenclosure[Attached]{Detailed response to Selection Criteria, Curriculum Vit\ae{}}


%-------------------------------------------------------------------------------
\begin{document}

% Print the header with above personal informations
% Give optional argument to change alignment(C: center, L: left, R: right)
\makecvheader[C]

% Print the footer with 3 arguments(<left>, <center>, <right>)
% Leave any of these blank if they are not needed
\makecvfooter
 {\today}
 {Dr Brian Ballsun-Stanton~~~·~~~Cover Letter and Selection Criteria Response}
 {\thepage}

% Print the title with above letter informations
\makelettertitle

%-------------------------------------------------------------------------------
%	LETTER CONTENT
%-------------------------------------------------------------------------------
\begin{cvletter}


I am a data scientist with over nine years designing, delivering, and securing technical solutions for academic and student research projects at the Macquarie University Faculty of Arts and UNSW Australia. My PhD (Philosophy of Data) explored how different people bridge different understandings of the nature of data, providing a transdisciplinary insight between Information Technology disciplines of data, development, and security with the needs and understandings of non-technical researchers. I will bring my strong technical and academic knowledge -- and the skills needed to translate between those domains -- to my teaching and research. My current role entails technical advice, database designs and solution-focused custom software support for academic and student research projects across Australian and International academic institutions. As part of that role, I am a CI on many category one and two grants. I am an internationally respected researcher in the Digital Humanities with sixteen peer-reviewed publications, commissioned reports, newspaper articles, book chapters, and conference presentations -- despite having worked primarily in professional rather than academic roles.

I am applying for a lateral promotion to Senior Lecturer in Cybersecurity Analysis as a natural next step to build on my academic and teaching capabilities. I am excited to contribute and enhance the cybersecurity program and stream in the Department of Security Studies and Criminology. I am confident that I can teach the department's cybersecurity curriculum and contribute my unique insights and technological approaches to our students. My research is fundamentally digitally enabled. Besides the study of big social media data in exploring online extremism, my research explores how to produce more FAIR and secure datasets for reproducible research results. I would be delighted to contribute to units on the ethics of cybersecurity as my explorations of the philosophy of technology can provide a novel contribution in that regard.

\lettersection{Expertise}

I am Chief Investigator in category one and two grants, contracts, and prizes across the humanities, social sciences, and security studies totalling over \$3,248,624. These have included being Lead Investigator on a high-impact, big data, investigation of violent extremism using tens of millions of posts on social media deploying computational data collection and analysis techniques using Machine Learning and Natural Language Processing. I was responsible for securing, collecting, and analysing the highly sensitive data per the data management plan I designed. I also am the Technical Director for a field-data collection project, delivering 64+ field data collection modules in archaeology, geoscience, ecology, oral history, and other domains. These projects demonstrate my ability to provide reproducible, robust, and secure data collection methodologies and to collaborate with researchers across disciplines. I remain an active member of The Carpentries, Instructor Trainer, and Instructor in good standing. I have co-supervised two MRES students using technology as a fundamental component of their research. 

I also have strong industry relationships: I developed a partnership with Ubisoft that led to Macquarie University's engagement on the Google Arts and Culture Platform, using AI to assist researchers in translating Ancient Egyptian Hieroglyphs. I also have extensive long-term relationships with software companies like Google and Twitter and governmental organisations like the Department of Communities and Justice. 

As the Faculty of Arts' Data Scientist (Solutions Architect), I draw on my academic and practical cyber capabilities (PhD in the Philosophy of Data, BS and MS in Information Technology) to drive and secure our digitally-enabled research plans for faculty and postgraduate research. I have delivered twenty software projects collaborating with researchers and professional staff across the university. The successes of these projects demonstrate my excellent technical capabilities, ability to understand and translate the needs of my colleagues into secure technical solutions, and to incorporate these projects into practical and academic outputs.

\lettersection{Research}

In 2019, I commenced a collaboration with Dr Julian Droogan and Lise Waldek to investigate online right wing-extremism through the systematic collection and mixed-method qualitative and quantitative analysis of very large social media datasets. Our collaboration has generated three category two grants from the NSW government and internationally from the United States Institute of Peace. My contribution has combined programming and technical expertise with cybersecurity and systems administration underpinned by a strong foundation of philosophical expertise. I developed the systems, security, databases, programs, data collection scrapers (Gab) and API driven data extraction programs (Twitter, YouTube). I performed data analysis at scale through the application of thorough database design principles with Machine Learning approaches to Natural Language Processing to develop quantifiable and actionable datasets for subsequent evidence-based publication.

As a result of this research, I was able to leverage my security insights into shaping the university's Data Management Standards in addition to providing education around those plans in the form of workshops and an online module (featured to strong acclaim in PICT8012) developed in collaboration with Research Ethics and Integrity. As a senior lecturer in the department, I will be excited to be the department's primary advocate on faculty and university ethics committees. 

My philosophical expertise provided analytical rigour to the project, allowing clear demarcations between predictive hypothesis-generating analysis techniques and postdictive hypothesis testing techniques, leveraging the power of persuasive case studies to fund a larger and more rigorous exploration across the wider Australian population. We have presented this research to Harvard, NATO, the Minister for Counter-Terrorism and Corrections NSW, and to federal and parliamentary committees. We expect to continue this excellent track record of external research funding, impact, and engagement for years to come.

I am also the Technical Director for the prize-winning Field Acquired Information Management Systems Project. As Technical Director, I have been responsible for software design, sysadmin, DevOps, and client management since 2012. As a result of my design, we have deployed over sixty data collection modules to almost every continent. I have collaborated with researchers and teams across the world in many disciplines to design and deliver effective field data management capabilities. We are currently on an ARDC funded project round, and this project has purchased 0.4FTE of my Solutions Architect job. 

The ARC will shortly announce the results for a \$189,111 early childhood education Linkage grant that I am also a CI on -- providing data management, software design, and crucially a cybersecurity implementation for highly sensitive early childhood data. This grant has requested 0.2 FTE (in-kind) until 2024. I am essential in providing highly secure data collection infrastructure -- provisioning multi-factor authentication and communicating the security needs of the project to five hundred early childhood educators working in preschools across the country. Beyond that, I also will perform DevOps, Project Management, and System Administration, securing cloud-based analysis virtual machines and managing software requirements and design with the Research Software Engineers at AAO@MQ. This project demonstrates how I can provide a strong and secure technical capability in collaboration with non-technical researchers across the faculty. My infrastructure designs and support were specifically praised as a critical part of project feasibility in the ARC assessors' reports.

As a result of my research projects, I have strong relationships with Twitter and Google, facilitating postgraduate research using Twitter Data (two PhD students) and being invited to consult on Google's Machine Learning projects (one MRes student, co-supervision). These industry ties allow practical discussions of the tech industry's needs along with facilitating research for \textit{and about} the tech industry and its security practices. 

The demands of my professional role at MQ have limited my research opportunities. However, despite these challenges I have collaborated on sixteen peer-reviewed publications (two of which are accepted in-press for publication next year). I have also collaborated in data preparation and publication with multiple postgraduate students, two of whom are currently preparing their data for publication as a peer-reviewed output. I use Python, SQL, \LaTeX, and Natural Language Processing to facilitate these many research projects. I am also part of the Program Committee for the International Association for Computing and Philosophy -- which keeps me current on many of the ethical debates we have in computing and security. I also have been part of the program committee for RezBaz Sydney (a multi-university computing skills celebration for researchers and postgraduate students) and am the production editor for the Archipelagos Journal. 

\lettersection{Teaching}

I have extensive teaching experience. As the first instructor and instructor trainer of The Carpentries (an organisation dedicated to: `[Teaching] foundational coding and data science skills to researchers worldwide') at Macquarie University, I have led a group of staff and postgraduate volunteers to empower over five hundred researchers and postgraduate students to better utilise software analysis methods for reproducible research. I have also co-supervised two MRes students and one PhD student using digital methods to perform reproducible social sciences research. 

This trans-disciplinary digitally-enabled research approach demonstrates my commitment to the provision of excellence in cyber education. I enable our students to meet the rigorous demands of industry, requiring both an excellent technical basis that informs evidence-based insights. Technically, I will expect our graduates to be comfortable with the command line and basic programming techniques such that they can effectively communicate its limitations to clients and mediate between those clients and the necessary technical work. This cross-disciplinary approach is necessary for the ability to communicate their actionable insights to a sceptical (and terrified) non-technical audience. Fundamental to my success is my ability to bridge technical and non-technical domains and to provide higher education approaches that facilitate the acquisition of both skillsets -- something that industry has articulated their strong demands for. 

I am cultivating a group of postgraduate students and ECR colleagues focused on creating a proactive environment of software-based reproducible data analysis, bringing the techniques of Information Technology to support digitally-enabled research across the faculty. From ethics, security and data management plans, to data collection at scale and machined based analytical techniques, my work promotes world-class research techniques across the faculty. I am committed to fostering a vibrant ECR culture of researchers supporting researchers -- learning and sharing these software techniques to increase the impact and scale of all of our research.

I am excited to be able to convene PICT8048, 40, and 80. Fundamental to my skill-set is the ability to bridge the technical and humanities worlds -- serving as an advocate for security to non-technical researchers and implementing cybersecurity protocols to protect and enhance their research. I will support our students in using technical skills to understand and frame evidence and then further expect them to articulate that evidence to a non-technical audience. Fundamental to cybersecurity is the practice and assurance of using data as evidence for a persuasive story and actionable insights. I will be able to draw on my active research, technical, and philosophical backgrounds to be a role model of how students in the humanities can collaborate and elevate a technical partnership. My skill set can drive an exciting interdisciplinary cybersecurity program within the department, which, as my industry contacts remind me, there is real-world demand for graduates who can engage with both software and stakeholders.

I would be pleased to build on my experience with convening FOAR705, units in Rochester Institute of Technology's Information Technology and Security program, pedagogical instruction in The Carpentries and my considerable digitally-enabled research program to amplify and enhance the current cybersecurity program and develop its technical, ethical, and communicative approaches. I am uniquely situated to blend technology and theory for a cutting-edge degree suitable for industry and academic needs. 


\lettersection{What I would like from the role}

I am interested in applying at rank C, step 1 -- a lateral move from my current HEW9.2. Migrating my current funded research responsibilities will require a gradual transition into a full-time teaching load. However, with some of my funded research, I can certainly use that to buy out portions of my teaching and marking.

I am excited to take on postgraduate teaching in cybersecurity, as this is one place where my professional role has stagnated. I also look forward to engaging in service and leadership in the Faculty: I would be a natural member of the faculty eResearch and human research ethics committees, as well as providing my experience with category one and two funding, research design, and digitally-enabled research support to the department level research committee. I would also like to explore how micro-credentialling and the Carpentries workshop techniques can impact Technology-Enabled Teaching and Learning and could promote these tested techniques on that committee. 

I would also like to continue my involvement in The Carpentries. As one of the senior trainers in Australia, I am trying to build the Australian Carpentries Community across many universities here. As part of that, I would like to run four local workshops a year (two instructor-training, two software workshops) for our department, the faculty, and the university as a whole. 




\end{cvletter}


%-------------------------------------------------------------------------------
% Print the signature and enclosures with above letter information

%\letterenclosure[~]{}

\makeletterclosing

\end{document}
