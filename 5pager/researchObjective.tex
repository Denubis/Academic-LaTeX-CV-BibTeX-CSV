\cvsection{Personal Profile}

Dr Brian Ballsun-Stanton is leading Macquarie University's Faculty of Arts understanding, policy, teaching, and research initiatives around the techniques and appropriate uses of Generative AI. He has developed the university's "Guidance Note: Using Generative Artificial Intelligence in Research" and has deployed LLMs in various courses, including those in Languages, Security Studies, and Law, each with unique aspects such as AI-optional and AI-required assessments, cross-model critical evaluation, and simulations. He is also teaching classes in the effective use and techniques of generative AI to undergraduate and postgraduate students.

Brian has also conducted invited workshops and presentations on LLMs for staff, postgraduate students, and external stakeholders beyond the university. He has advised staff and postgraduate students on API use, research ethics involving LLMs, and automated grading. Additionally, he has presented the "Introduction to LLM" workshop to early-career researchers and postgraduate students across Australia, the United States, and Europe on more than 20 occasions.

He has more than fifteen years of experience as a data scientist, educator, sysadmin, and data security specialist, designing and delivering technical solutions for academic and student research projects at the Macquarie University Faculty of Arts and UNSW Australia. He has served as Chief Investigator in grants and projects across the humanities, social sciences, and security studies totaling over AUD\$3,514,323, and has led a big data investigation of violent extremism using tens of millions of social media posts. As a Technical Director, Product Owner, and DevOps/SysAdmin, for Electronic Field Notebooks PTY LTD and the FAIMS Project, he has delivered more than seventy field data collection modules since 2013. Brian is also an internationally respected researcher in the Digital Humanities with eighteen peer-reviewed publications, commissioned reports, newspaper articles, book chapters, and conference presentations.



