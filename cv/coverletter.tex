%!TEX TS-program = xelatex
%!TEX encoding = UTF-8 Unicode
% Awesome CV LaTeX Template for Cover Letter
%
% This template has been downloaded from:
% https://github.com/posquit0/Awesome-CV
%
% Authors:
% Claud D. Park <posquit0.bj@gmail.com>
% Lars Richter <mail@ayeks.de>
% Brian Ballsun-Stanton (brian.ballsun-stanton@mq.edu.au)
%
% Template license:
% CC-BY-SA 4.0 International
%


%-------------------------------------------------------------------------------
% CONFIGURATIONS
%-------------------------------------------------------------------------------
% A4 paper size by default, use 'letterpaper' for US letter
\documentclass[11pt, a4paper]{awesome-cv}

% Configure page margins with geometry
\geometry{left=1.4cm, top=.8cm, right=1.4cm, bottom=1.8cm, footskip=.5cm}

% Specify the location of the included fonts
\fontdir[fonts/]

% Color for highlights
% Awesome Colors: awesome-emerald, awesome-skyblue, awesome-red, awesome-pink, awesome-orange
%         awesome-nephritis, awesome-concrete, awesome-darknight
\colorlet{awesome}{awesome-red}
% Uncomment if you would like to specify your own color
% \definecolor{awesome}{HTML}{CA63A8}

% Colors for text
% Uncomment if you would like to specify your own color
% \definecolor{darktext}{HTML}{414141}
% \definecolor{text}{HTML}{333333}
% \definecolor{graytext}{HTML}{5D5D5D}
% \definecolor{lighttext}{HTML}{999999}

% Set false if you don't want to highlight section with awesome color
\setbool{acvSectionColorHighlight}{false}

% If you would like to change the social information separator from a pipe (|) to something else
\renewcommand{\acvHeaderSocialSep}{\quad\textbar\quad}


%-------------------------------------------------------------------------------
%	PERSONAL INFORMATION
%	Comment any of the lines below if they are not required
%-------------------------------------------------------------------------------


\name{Dr}{Brian}{Ballsun-Stanton}
\position{Solutions Architect (Digital Humanities)}


\address{Faculty of Arts, Macquarie University, NSW 2109}

\mobile{(+61) 479 178 749}


% \address{First Address \\ Second Address}
% \mobile{(+61) 111 111 111}

\email{brian.ballsun-stanton@mq.edu.au}
\orcid{0000-0003-4932-7912}
\github{denubis}
\homepage{https://osf.io/dza9b/}
% \gitlab{gitlab-id}
% \stackoverflow{SO-id}{SO-name}
% \twitter{@twit}
% \skype{skype-id}
% \reddit{reddit-id}
% \medium{madium-id}
\googlescholar{gc0PEWQAAAAJ}{}
%% \firstname and \lastname will be used
% \googlescholar{googlescholar-id}{}
% \extrainfo{extra informations}

%-------------------------------------------------------------------------------
%	LETTER INFORMATION
%	All of the below lines must be filled out
%-------------------------------------------------------------------------------
% The company being applied to
\recipient
 {Office of the Pro Vice-Chancellor (Research Performance)}
 {Macquarie University\\Balaclava Rd\\Macquarie Park NSW 2109}
% The date on the letter, default is the date of compilation
\letterdate{\today}
% The title of the letter
\lettertitle{Job Application for Research Data Architect, Data Science and eResearch}
% How the letter is opened
\letteropening{Dear Professor Barnier,}
% How the letter is closed
\letterclosing{Sincerely,}
% Any enclosures with the letter
\letterenclosure[Attached]{Detailed response to Selection Criteria, Curriculum Vit\ae{}}


%-------------------------------------------------------------------------------
\begin{document}

% Print the header with above personal informations
% Give optional argument to change alignment(C: center, L: left, R: right)
\makecvheader[C]

% Print the footer with 3 arguments(<left>, <center>, <right>)
% Leave any of these blank if they are not needed
\makecvfooter
 {\today}
 {Dr Brian Ballsun-Stanton~~~·~~~Cover Letter and Selection Criteria Response}
 {\thepage}

% Print the title with above letter informations
\makelettertitle

%-------------------------------------------------------------------------------
%	LETTER CONTENT
%-------------------------------------------------------------------------------
\begin{cvletter}


I am writing to apply for the position of Research Data Architect in the Data Science and eResearch unit. My experience as the Solutions Architect in the Faculty of Arts, the Carpentries Trainer for the university, and facilitating Data Management Standards with Research Integrity, aligns well with this position's selection criteria. Besides my professional work supporting researchers at the university, I am currently an investigator on three external grants. 

\lettersection{About Me}

I have a PhD in Philosophy from UNSW Australia, a BS (with Highest Honors) and an MS in Information Technology from Rochester Institute of Technology. I have worked as the Technical Director for the FAIMS Project since 2012, deploying geospatial field data collection workflows to over 50 projects worldwide. My research using Natural Language Processing and Python to analyse hundreds of thousands of posts on social media was presented to the state Counter Terrorism Committee, the Minister for Police and Emergency Services and I will be presenting it to the Commonwealth in August. 

For the last three years, I have worked to design and implement technical solutions for researchers' challenges across the Faculty of Arts. I maintained virtual machines on NCI Tenjin, using them for Natural Language Processing, image processing of Ancient Egyptian tomb art, and web-scraping many data sources. I have created a tool to convert 1000+ citations into \href{https://onwork.edu.au}{onwork.edu.au}. I facilitated interactions between the Faculty of Arts, Google Arts and Culture, and Ubisoft -- leading to the launch of the `Fabricius' AI tool. I trained hundreds of researchers and students in data management, cleaning, and version control. I have written software for many projects to collect, manipulate, and analyse data. Data and papers from my projects have been published for peer review on the Open Science Framework. 


\lettersection{Why does Macquarie want me?}

As Research Data Architect for the University, I will leverage my technical expertise and in-depth organisational knowledge to ensure Macquarie becomes a champion of research integrity and impact. My strong commitment to publishing FAIR data and my proven efficiencies in data collection, analysis, publication, archiving and reuse will ensure data in our research outputs across the university are, `As open as possible, and as closed as necessary.' Our data will be publishable, citable, and compliant with the transparency and openness requirements of funding bodies and journals.

My extensive experience in data manipulation is fundamentally interdisciplinary and covers the entire software development life-cycle. I can move between questions from academic researchers as well as those from university staff. Once problems are articulated, I can scope and elaborate technologies to ensure we have found solutions that are responsive to the problems at hand. I then can implement proofs-of-concept myself, or liaise with developers to create production-ready systems and testing regimes to ensure that the solutions \textit{solve} the problem at hand. 

As a Digital Humanist, I bridge the academic and technical worlds, bringing key skills and problem-solving methods from both into play. I have been a Project Manager for large software projects. I have won grants on the strength of my data manipulation. I have designed, elaborated, programmed, tested, and delivered small projects. I have used the Agile software development lifecycle to deliver a project into worldwide use. I am a professional data architect and an excellent fit for this position.

% \makeletterclosing

\newpage


\lettersection{Detailed responses to selection criteria}

\lettersubsection{Outreach, training, and support}

Since 2017 I have been actively engaged in outreach, training, and support of my colleagues at Macquarie University. I am an active Trainer and Instructor in The Carpentries. I have developed Carpentries workshops for webscraping and active data management. I have delivered data collection and analysis solutions to colleagues across the Faculty of Arts in Security Studies, International Studies, and Philosophy. I have deployed field data collection platforms and collaborated with academics around the world. I have worked with OpenStack on NCI Tenjin and deployed highly sensitive research infrastructure to AWS. I mostly work in SQL, Python, \LaTeX{}, and Jekyll static pages served over Github Pages. I have a proven capacity to develop effective research technology training.


\textbf{`Active' Data Storage:} 
\begin{letteritems}
\item {In April 2020, I worked with AARnet's Dr Sara King to write and deliver a workshop on Active Data Management using cloudstor. This was met with very positive feedback. One attendee, in the final session said, `[A]mazing retention - can't believe everyone is still here at the end - that is testament enough to how engaging you guys were and
the excellent communication skills. Brian and Sara were amazing at volleying between each other and delegating and sharing
tasks… it was a pleasure to watch something that worked so functionally!'}
\item {I am a user acceptance tester and frequent contributor on AARnet's slack.}
\item {I currently have peer review blinded data published on the Open Science Framework, along with preprints, demonstrating best practices to sharing research data -- and placing me in a strong position to help others achieve this same outcome.}
\end{letteritems}

\textbf{The Carpentries:}
\begin{letteritems}
\item {I am on The Carpentries Trainer Leadership group. I am a badged Carpentries Instructor Trainer, with plans to run online training in July -- August with Dr Darya Vanichkina from the University of Sydney, September with new trainers from the ARDC, and November with trainers from Pawsey.}
\item {I will be Lead Instructor in a July online workshop using bash and Python. I am planning an August workshop on `From a Spreadsheet to a Database.'}
\item {I have submitted a meta-workshop demonstrating using Carpentries workshop development pedagogy to make a workshop to the online CarpentryCon in August.}
\end{letteritems}

\textbf{Developing Research Technology Training:} 
\begin{letteritems}
\item {I was co-convener and co-creator of the Digital Humanities unit for the Faculty of Arts, teaching MRES students how to apply the software development lifecycle to their research projects, and how to find, program, and implement technology to solve their research problems.}
\item {I have rewritten the Library Carpentry Webscraping lesson (which Intersect Australia uses in preference to the original), am the maintainer for the `From a Spreadsheet to a Database' Software Carpentry lesson, and wrote with an AARnet colleague `Introduction to Active Data Management' as a Data Carpentry lesson early this year.}
\item {I am collaborating with the Research Integrity Secretariat on an ARDC grant-funded online course introducing FAIR data and the new Data Management Standards across the university.}
\end{letteritems}

%Experience developing research technology training curricula and instructional materials (desirable).

\textbf{Promote good practice:} 
\begin{letteritems}
\item {Since 2012, I have worked successfully to persuade and provide archaeologists and field researchers a mobile platform designed for highly structured and rigorous data collection, in a form ready for publication.}
\item {I have run multiple workshops on transparency in research data. I have a paper, forthcoming, which speaks to the need for preregistration in Archaeological data management.}
\item {My far-right research grant served as a demonstration testbed for the proposed Data Management Standards -- demonstrating the mechanisms, security, and backup strategies, which can apply to even highly sensitive research.} 
\end{letteritems}

%Experience working with academics or research students to promote and support good practice in research data management and use of research technologies (essential).

\textbf{Deploy solutions:} 
\begin{letteritems}
\item {I have deployed over 64 field data collection modules and data exporters since 2012 as the Technical Director for the FAIMS project, helping researchers around the world collect better data while in the field.}
\item {I have written and deployed over 20 other projects since the start of my job as Solutions Architect for the Faculty of Arts in 2017. Of specific mention:} 
\begin{enumerate} 
	\item {The custom post collection and analysis Natural Language Processing infrastructure for the Far-Right Extremism grant collected and analysed hundreds of thousands of posts for Security Studies;}
	\item {Webscraping more than a thousand citations in Python to generate \href{https://onwork.edu.au}{onwork.edu.au} for Philosophy}
	\item {Programming a Trove to Endnote parser harvesting hundreds of old newspaper citations, their metadata, and their pdfs into an Endnote datastore for Modern History}
	\item {Developing analysis software which uses MeCAB to parse, tokenise, and lemmatise Japanese Insurance Advertisements to support research in International Studies.}
\end{enumerate}
\end{letteritems}

%Practical experience working with researchers to develop and deploy technology solutions (essential).

\textbf{Data analysis from multiple domains:} 

\begin{letteritems}
\item {My primary strengths are requirements elicitation, relational database design, data loading, and analysis. I have applied these strengths to geospatial problems using GIS infrastructure in PostgreSQL and SQLite.} 
\item {I have also used Natural Language Processing to prepare material for both quantitative corpus analysis using log-likelihood statistical techniques and graphing trends in activity over time. I have supported qualitative analysis using ggplot and matplotlib to
generate wordclouds, lexical dispersion plots and deidentified excerpts.} 
\item {I have successfully supported data analysis of relational and tabular data across multiple projects by publishing it to Con\TeX{}t rendered PDFs, processing it into YAML and rendering it into websites through Jekyll, creating summary reporting spreadsheets, reformulating it for IGSN submission, and presenting it with photos in KML files.}
\item {Supporting university operations and outside of a purely academic research context, Prof Brawley and faculty general manager Neil Durrant have used my expertise, in different projects. They needed me to extract, transform, and load course and unit data from multiple university data warehouses and sources into an OLTP database to answer questions about course feasibility and unit complexity. As part of this unit analysis, I built visualisations of unit requirement relationships via directed graphs using Graphviz.}
\end{letteritems}


% Experience with data analysis techniques from multiple domains (essential).

\textbf{Scripting and development languages:} 
\begin{letteritems}
\item {I am expert in the use of SQL (SQlite, Spatialite, PostgreSQL, Oracle, PL/SQL), Python (NLTK, Spacy, Matplotlib, ggplot, MechanicalSoup, BeautifulSoup, LXML, Jupyter), bash (SSH, HPC using GNU Parallel), XML, XSLTs, \LaTeX, Con\TeX{}t, markdown (pandoc and Jekyll), HTML, Javascript, and CSS to collect, analyse, and present data.} 
\item {I have experience with node.js, PHP, Ruby, R, Java, and C++. }
\end{letteritems}

% Experience with scripting and development languages (essential).

\textbf{Modern research computing environments:} 
\begin{letteritems}
\item {Most of my significant compute load is performed on NCI Tenjin virtual machines using the OpenStack virtualisation platform (as NeCTAR also uses). I run Linux for preference on all of my computers as my primary operating system and have considerable experience in system administration using the bash shell.} 
\item {I have experience deploying containers to Docker repositories, using docker images, and using YAML description files to deploy AWS setups.}
\item {I use and teach modern collaborative distributed version control systems (Git, Github, GitLab, Bitbucket) for my academic and professional research, as well as for work I do with The Carpentries.}
\item {I use WebDAV (owncloud), S3, and Google Cloud Storage as the endpoints of automated storage interactions. I have used GNUPG and Yubikey Hardware Storage Modules to read and write secure encrypted data across multiple computers. }
\end{letteritems}

% Experience with modern research computing environments (essential).

\lettersubsection{Infrastructure}

I used the Agile methodology and the Rational Unified Process with multiple teams of programmers at external contractors and inside the university to deliver the FAIMS android app for use around the world. I teach colleagues how to use and backup to Cloudstor and have been the primary developer-operations support for the FAIMS cloud infrastructure. 


\textbf{Software Development Lifecycles:} 
\begin{letteritems}
\item {I was Data Architect and the Technical Director for the FAIMS Project, successfully using agile software development lifecycles and the Rational Unified Process to deliver a server and android application used in production in universities around the world, using Atlassian's JIRA and Confluence to track requirements and deliverables across multiple vendors.}
\item {I am responsible for delivering scoping and elaboration results on the FAIMS3 ARDC grant. I was the Lead Investigator for a documentation grant from NeCTAR for FAIMS2, and am the editor and product owner of the `User to Developer Guide' for FAIMS.}
\item {My PhD dissertation entailed adapting requirement elicitation workflows (Data Flow Diagrams) for philosophical investigation. I have taught Waterfall and Agile development methodologies along with requirements analysis and generation at Macquarie University and Rochester Institute of Technology.}
\end{letteritems}

% Experience with systems development lifecycle and software development practices, including analysis, design, testing, documentation, delivery, operations, and evaluation, as well as project management and associated tools (essential).

\textbf{Research Technology Platforms:} 
\begin{letteritems}
\item {I was responsible for finding and managing deployment of the open-source Limesurvey research survey platform at Macquarie.}
\item {I have developed backup scripts and interaction protocols for Cloudstor (owncloud, webdav, S3) at Macquarie University and am responsible for considerable adoption of the platform in the Faculty of Arts for use as a collaboration tool and as an encrypted backup location.}
\item {I have deployed over 50 field data collection modules as the technical director of the FAIMS Project. As part of that project, I supervised FAIMS' development and deployment of the Heurist platform, and was system administrator) for the deployment of the tDAR data repository.}
\item {I have run workshops demonstrating the Open Science Framework and have some of my research on that repository.}
\end{letteritems}

% Experience deploying and operating research technology platforms and services, such as data repositories, ‘active’ data storage and collaboration platforms, or data capture systems (desirable).

\lettersubsection{Governance, policy and process}

Since joining Macquarie in 2015, I have successfully collaborated with colleagues and staff across the university to support and enhance their research. I have helped to formulate the Data Management Standards. I have served on the Resbaz Sydney organising committee and lead Carpentries instruction at that conference. I have lead Macquarie into partnerships with Google and Ubisoft. I have built relationships with the ARDC, AARnet, and AAF. 

\textbf{Experience at Macquarie:} 

\begin{letteritems}
\item {I have worked at Macquarie University since January 2015, first as the Technical Director for the ARC-funded FAIMS project, and then from 2017, the Solutions Architect (Digital Humanities) for the Faculty of Arts.}
\item {In these roles, I have worked closely with colleagues in the Faculty of Arts and have developed excellent and productive relationships with colleagues in the Faculty of Science and Engineering as well as Central IT, and Legal Services.}
\item {I have developed excellent rapport and a strong working relationship with the Research Integrity Secretariat.}
\end{letteritems}

% Experience working in a university or other research organisation that would allow rapid mastery of the institutional environment at Macquarie (essential).

\textbf{Implementing and Reviewing Policy:} 
\begin{letteritems}
\item {I provided technical implementation instructions considerations while helping to prepare the new Data Management Standard for Macquarie due to the NHMRC 2019 update.}
\item {I provided time-critical guidance to the Faculty of Arts HREC committee as well as the Research Integrity Secretariat when reviewing the privacy and records act implications of Zoom use for sensitive academic research during the COVID-19 crisis.}
\end{letteritems}

% Experience developing, implementing, and reviewing technology-related governance, policy, and processes at a large organization (essential).

\textbf{Engaging with stakeholders:} 
\begin{letteritems}
\item {As FAIMS' Technical Director, I elicited and integrated requirements to design a field data collection platform that could be used across the world, and then to ensure that the modules we created delivered on my design.} 
\item {As Solutions Architect for the Faculty, I served on Macquarie's Research Data and Compute committees, representing the needs of the Faculty of Arts' more esoteric projects to the committees. The success of these projects highlights my effective engagement with diverse stakeholders from both Faculty and University.}
\item {I was able to build on this engagement with my work for Professor Brawley as Data Architect for Stage 1 of his Course Health Check system -- drawing on data flows from many parts of the university to demonstrate that the answers he wanted from the system were possible.}
\item {As Macquarie's Carpentries Trainer, I have been responsible for working with eResearch to select, train, and motivate three years of instructors to participate in the Carpentries at Macquarie and in Sydney.}
\end{letteritems}

% Experience with successfully engaging with a diverse group of stakeholders within an institution, research facility or research group that spans multiple institutions (essential).

\textbf{Experience with leadership:} 
\begin{letteritems}

\item {As Technical Director for FAIMS, I was responsible for implementing and setting strategic technical priorities for the FAIMS platform, managing external programming teams and internal programmers, and managing expectations of the rest of the leadership team as a result of development goals and realities.}
\item {I am a member of the Research Bazaar Sydney's organising committee, making sure the yearly conference's training program is coherent, inclusive, and attractive.}
\item {I was invited to be one of the founding members of The Carpentries Trainer Leadership group, designing and setting policy for Carpentries Trainers around the world. My current priority is to ensure regional specificity for volunteer workload and support strategies, to better align with requirements and expectations placed on postgraduate students and university staff outside of the United States. }
\item {I am working with colleagues across Australia to build a cohesive and specific community for The Carpentries and its trainers in Australia.}
\item {I instigated and developed the first partnership between a full university and Google Arts and Culture. I successfully persuaded the Faculty of Arts, acting on behalf of the university, to engage with this platform. The entire university will benefit from this unique opportunity to deliver research engagement.}
\end{letteritems}

% Experience in a team, department, faculty, or organisational leadership role (desirable).

\textbf{Experience with national organisations:} 
\begin{letteritems}
\item {During my work as Software Architect for the Faculty of Arts, I have worked with AARnet to test and refine their Cloudstor offerings.}
\item {I have worked with the Australian Access Federation, to integrate it with Macquarie's Limesurvey instance, to implement specific security patterns as part of my highly sensitive far-right research, and to report security bugs and issues.}
\item {I have worked with the ARDC, not only for the FAIMS and Introducing FAIR grants but as part of their community of practice, the `Research Data Roundtable.'}
\end{letteritems}


% Experience working in or closely with (national) research organisations/facilities or similar (desirable).

\lettersubsection{Technology-enabled research}

I have a significant research profile spanning six grants and eleven peer-reviewed publications, despite spending only two of eight post-PhD years in an academic role. I have enabled innovative research in the Faculty of Arts, providing a scope and consistency impossible without technological support. I have delivered and supported field-research collection solutions around the world. I have created impact and engagement for Macquarie's research, having my Far-Right research being presented to state ministers, and creating engagement on Google Arts and Culture and Fabiricus. 


\textbf{Experience in data-intensive collaborative research:} 
\begin{letteritems}
\item {My research career has focused around preparing, collecting, analysing, and publishing data in deep collaboration with researchers around the world.}
\item {In FAIMS, I deployed over 50 projects for teams of researchers worldwide, from Archaeology and Ecology to Oral History and community biological sampling. I have served as emergency on-call tech support to many of those projects and travelled to the field to support FAIMS in Israeli excavations at Khirbet el Rai.} 
\item {I was responsible for the extensive collection and analysis of social media posts for my Right-Wing Extremist grant examining six different platforms and an MRES student project studying Facebook.} 
\item {I work with a team of academics from Philosophy to make sure the citation-to-website pipeline for onwork.edu.au is functional.}
\end{letteritems}

% Experience participating in data-intensive, collaborative academic research, particularly multi-disciplinary research involving teams of researchers (essential).

\textbf{Experience publishing:} 
\begin{letteritems}
\item {I have been CI on three grants (Far-Right Extremism, FAIMS 3, and the FAIMS 2 documentation grant).}
\item {I have achieved an h-index of five on eleven peer-reviewed publications, my dissertation, and one technical report with CSIRO. }
\item {I have presented my research, in person and through confidential reports, at domestic and international academic conferences and to a range of government stakeholders including the NSW Counter-Terrorism committee and the Minister for Police and Emergency Service.}
% Experience publishing academic research and winning grants or other external funding (essential).
\end{letteritems}

\lettersection{What do I want from Macquarie?}


The expectations of this position fit my career objectives and skills exactly. The position allows me to make sure my research in Digital Humanities has a real-world impact: innovating, shaping, and improving the `Digital Transformation' of Macquarie in a `digital first world'. I look forward to interviewing for this position and presenting my research to my colleagues. Thank you for your consideration.

\end{cvletter}


%-------------------------------------------------------------------------------
% Print the signature and enclosures with above letter information

\letterenclosure[~]{}

\makeletterclosing

\end{document}
